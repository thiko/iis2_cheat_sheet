\documentclass[a4paper,11pt,numbers=noendperiod,bibliography=totocnumbered,listof=totocnumbered,abstracton]{scrreprt}
\usepackage[ngerman]{babel}
\usepackage[utf8]{inputenc} %windows latin1
\usepackage[babel,german=quotes]{csquotes}
\usepackage[T1]{fontenc}
\usepackage{lmodern}% Schrift ist bei allen modernen TeX-Distributionen dabei und der Standard-T1-Schrift deutlich ¸berlegen 
\usepackage{setspace}						% Paket um Absätze zu definieren
\usepackage[left=4cm,right=2cm,top=2.5cm,bottom=2.5cm,includeheadfoot,headheight=1.3\baselineskip]{geometry}% Paket für Seitenlayout

\usepackage{amsmath}						% AMS Pakete für neue Mathe-Umgebungen und -Zeichen
\usepackage{amsfonts}
\usepackage{amssymb}		% Paket f¸r Symbole. ‹bersicht: http://amath.colorado.edu/documentation/LaTeX/Symbols.pdf

\usepackage{blindtext} % Beispieltext

\usepackage{wrapfig} % Bild neben Text
\usepackage{multicol} % Mehrspaltenmodus
\usepackage{subfigure} % Bilder nebeneinander
\usepackage{stmaryrd} % Widerspruch Blitze


\renewcommand{\arraystretch}{1.5} % General space between rows (1 standard)
\setlength{\tabcolsep}{5pt} % General space between cols (6pt standard)

%% Tiefe des Inhaltsverzeichnisses
\setcounter{tocdepth}{3}


%%%%%%%%%%% Seitenzahlen rechts positionieren %%%%%%%%%%%%%%%%%%%
\usepackage{scrpage2}
\pagestyle{scrheadings}
\clearscrheadings
\clearscrplain
\ofoot[\pagemark]{\pagemark}
\cfoot{}
%%%%%%%%%%%%%%%%%%%%%%%%%%%%%%%%%%%%%%%%%%%%%%%%%%%%%%%%%%%%%%%%%

%%%%%%%%%%% Fuflnoten durchlaufend nummerieren %%%%%%%%%%%%%%%%%%%
\usepackage{remreset}
\makeatletter\@removefromreset{footnote}{chapter}\makeatother
%%%%%%%%%%%%%%%%%%%%%%%%%%%%%%%%%%%%%%%%%%%%%%%%%%%%%%%%%%%%%%%%%


%%%%%%%%%%% Abk¸rzungsverzeichnis erstellen %%%%%%%%%%%%%%%%%%%%%
\usepackage[german,intoc]{nomencl}
% Befehl umbenennen in abk
%\let\abk\nomenclature
% Deutsche ‹berschrift
\renewcommand{\nomname}{Abk¸rzungsverzeichnis}
% Punkte zw. Abk¸rzung und Erkl‰rung
\setlength{\nomlabelwidth}{.20\hsize}
\renewcommand{\nomlabel}[1]{#1 \dotfill}
% Zeilenabst‰nde verkleinern
\setlength{\nomitemsep}{-\parsep}
\makenomenclature
%%%%%%%%%%%%%%%%%%%%%%%%%%%%%%%%%%%%%%%%%%%%%%%%%%%%%%%%%%%%%%%%%

\usepackage{titleref} % um auf Titel (‹berschriften) zu referenzieren
\usepackage{hyperref} % Links auf Inhaltsverzeichnis und Verweise verteilen...
\usepackage{url} % f¸r \url{http://www}, Option hyp erlaubt auch Umbruch nach "-"

%%%%%%%%%%%%%%%%%%%%%%%%%%%%%%%%%
% The following is needed in order to make the code compatible
% with both latex/dvips and pdflatex.
\ifx\pdftexversion\undefined
\usepackage[dvips]{graphicx}
\else
\usepackage[pdftex]{graphicx}
\DeclareGraphicsRule{*}{mps}{*}{}
\fi
%%%%%%%%%%%%%%%%%%%%%%%%%%%%%%%%%

%%TABELLEN
\usepackage{booktabs}

\onehalfspacing % Anderthalbzeilig
\setlength{\parskip}{10pt plus 4pt minus 2pt}
\parindent0pt

% COMMANDS
\newcommand{\homo}{Homomorphismus }
\newcommand{\homos}{Homomorphismen }
\newcommand{\epi}{Epimorphismus }
\newcommand{\epis}{Epimorphismen }
\newcommand{\mono}{Monomorphismus }
\newcommand{\monos}{Monomorphismen }
\newcommand{\syssig}{$Sys(\Sigma)$ } 
\newcommand{\prop}{Proposition }
\newcommand{\defi}{Definition }
\newcommand{\coro}{Corollary }
\newcommand{\lem}{Lemma }

%

